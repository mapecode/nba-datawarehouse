%%%%%%%%%%%%%%%%%%%%%%%%%%%%%%%%%%%%%%%%%
% Daily Laboratory Book
% LaTeX Template
% Version 1.0 (4/4/12)
%
% This template has been downloaded from:
% http://www.LaTeXTemplates.com
%
% Original author:
% Frank Kuster (http://www.ctan.org/tex-archive/macros/latex/contrib/labbook/)
%
% Important note:
% This template requires the labbook.cls file to be in the same directory as the
% .tex file. The labbook.cls file provides the necessary structure to create the
% lab book.
%
% The \lipsum[#] commands throughout this template generate dummy text
% to fill the template out. These commands should all be removed when 
% writing lab book content.
%
% HOW TO USE THIS TEMPLATE 
% Each day in the lab consists of three main things:
%
% 1. LABDAY: The first thing to put is the \labday{} command with a date in 
% curly brackets, this will make a new page and put the date in big letters 
% at the top.
%
% 2. EXPERIMENT: Next you need to specify what experiment(s) you are 
% working on with an \experiment{} command with the experiment shorthand 
% in the curly brackets. The experiment shorthand is defined in the 
% 'DEFINITION OF EXPERIMENTS' section below, this means you can 
% say \experiment{pcr} and the actual text written to the PDF will be what 
% you set the 'pcr' experiment to be. If the experiment is a one off, you can 
% just write it in the bracket without creating a shorthand. Note: if you don't 
% want to have an experiment, just leave this out and it won't be printed.
%
% 3. CONTENT: Following the experiment is the content, i.e. what progress 
% you made on the experiment that day.
%
%%%%%%%%%%%%%%%%%%%%%%%%%%%%%%%%%%%%%%%%%

%----------------------------------------------------------------------------------------
%	PACKAGES AND OTHER DOCUMENT CONFIGURATIONS
%----------------------------------------------------------------------------------------

\documentclass[idxtotoc,hyperref,openany]{labbook} % 'openany' here removes the gap page between days, erase it to restore this gap; 'oneside' can also be added to remove the shift that odd pages have to the right for easier reading

\usepackage[ 
  backref=page,
  pdfpagelabels=true,
  plainpages=false,
  colorlinks=true,
  bookmarks=true,
  pdfview=FitB]{hyperref} % Required for the hyperlinks within the PDF
  
\usepackage{booktabs} % Required for the top and bottom rules in the table
\usepackage{float} % Required for specifying the exact location of a figure or table
\usepackage{graphicx} % Required for including images2
\usepackage{lipsum} % Used for inserting dummy 'Lorem ipsum' text into the template

\newcommand{\HRule}{\rule{\linewidth}{0.5mm}} % Command to make the lines in the title page
\setlength\parindent{0pt} % Removes all indentation from paragraphs

%----------------------------------------------------------------------------------------
%	DEFINITION OF EXPERIMENTS
%----------------------------------------------------------------------------------------

\newexperiment{example}{This is an example experiment}
\newexperiment{example2}{This is another example experiment}
\newexperiment{example3}{This is yet another example experiment}
\newexperiment{table}{This shows a sample table}
%\newexperiment{shorthand}{Description of the experiment}

%---------------------------------------------------------------------------------------

\begin{document}

%----------------------------------------------------------------------------------------
%	TITLE PAGE
%----------------------------------------------------------------------------------------

\frontmatter % Use Roman numerals for page numbers
\title{
\begin{center}
	\centering
\HRule \\[0.4cm]
{\Huge \bfseries Lab Book \\[0.5cm] \Large Bases de Datos Avanzadas}\\[0.4cm] % Degree
\HRule \\[1.5cm]
\end{center}
}
\author{\Huge Mario Pérez Sánchez-Montañez \\ \\ \Huge David Camuñas Sánchez \\ \\ \Huge Francesco Zingariello \\[2cm]} % Your name and email address
%\date{Beginning 6 February 2012} % Beginning date
\maketitle

\tableofcontents

\mainmatter % Use Arabic numerals for page numbers

%----------------------------------------------------------------------------------------
%	LAB BOOK CONTENTS
%----------------------------------------------------------------------------------------

% Blank template to use for new days:

%\labday{Day, Date Month Year}

%\experiment{}

%Text

%-----------------------------------------

%\experiment{}

%Text

%----------------------------------------------------------------------------------------
\labday{Martes, 11 de Febrero}
\experiment{Elección de la tecnologías}
Se ha decidido usar como tecnología \textit{Python} debido a la multiples librerías que posee relacionadas con el ámbito del análisis de datos.
\experiment{Elección del motor de base de datos}
Se ha elegido como motor de nuestra base de datos \textit{SQLite}, ya que soporta funciones SQL y podemos almacenar la base de datos completa en un solo archivo.

\labday{Jueves, 13 Febrero 2020}

\experiment{Obtención de los datos para el \textit{DataWarehouse}}
Desarrollamos tres \textit{scripts} en \textit{Python} utilizando técnicas de \textit{web scrapping} para obtener datos de jugadores y equipos de \href{https://www.basketball-reference.com/leagues/NBA_2019.html}{\textit{Basketball Reference}}.


\experiment{Cambio en diseño del \textit{Data Warehouse}}
Decidimos que vamos a tener una tabla con todos los jugadores, con los 30 equipos y de \textit{rookies} (jugadores nóveles). Todo ello estará relacionado en una tabla central, siguiendo el modelo de estrella.


\labday{Viernes, 14 Febrero 2020}

\experiment{Primer versión de la base de datos}
Realizamos la construcción de la primera versión de la base de datos. Uno de los problemas que hemos encontrado es con los jugadores que han jugado en varios equipos en la misma temporada, ya que en la fuente que hemos usado se muestran datos en cada equipo donde han jugado y lo totales, identificando estos últimos como 'TOT' en la columna de equipo.\\

Hemos decido mantener esta estructura identificando las estadísticas totales con -1, para que en la base de datos no se utilice un id de un equipo existente.

\experiment{Primeras consultas en la Base de Datos}
Hemos realizado las primeras las consultas en la base de datos de nuestro Data WareHouse. 
Cuyo resultado ha salido satisfactoriamente.


\labday{Lunes, 17 Febrero 2020}

\experiment{Cambio en la obtención de datos}
Se ha realizado unos cambios en cuanto a la forma de obtener los datos. Estos cambios han afectado al número de \textit{scripts}. Ahora los datos se obtienen mediante un solo \textit{script} (en \textit{Python}), cuyo objetivo es obtener los datos mediante la técnica conocida como \textit{Web Scrapping}. Además para la creacíon de las tablas que compondrán nuestra base de datos, se utiliza otro nuevo \textit{script} (en \textit{SQL}).

\experiment{Realización de consultas}
Se han realizado una serie de consultas, cuyo objetivo es obtener la información necesaria de nuestra base de datos. Y así obtener información para poder llevar a cabo un estudio estadístico.
Este estudio se realizara mediante el uso de consultas (dicho anteriormente) y el uso de \textit{dataframes}.


%-----------------------------------------

%\experiment{example2} % Multiple experiments can be included in a single day, this allows you to segment what was done each day into separate categories
%
%\begin{figure}[H] % Example of including images
%\begin{center}
%\includegraphics[width=0.5\linewidth]{example_figure}
%\end{center}
%\caption{Example figure.}
%\label{fig:example_figure}
%\end{figure}
%
%%-----------------------------------------
%
%\experiment{example3}
%
%\lipsum[3-5]
%
%%----------------------------------------------------------------------------------------
%
%\labday{Friday, 26 March 2010}
%
%\experiment{table}
%
%\begin{table}[H]
%\begin{tabular}{l l l}
%\toprule
%\textbf{Groups} & \textbf{Treatment X} & \textbf{Treatment Y} \\
%\toprule
%1 & 0.2 & 0.8\\
%2 & 0.17 & 0.7\\
%3 & 0.24 & 0.75\\
%4 & 0.68 & 0.3\\
%\bottomrule
%\end{tabular}
%\caption{The effects of treatments X and Y on the four groups studied.}
%\label{tab:treatments_xy}
%\end{table}
%
%Table \ref{tab:treatments_xy} shows that groups 1-3 reacted similarly to the two treatments but group 4 showed a reversed reaction.
%
%%----------------------------------------------------------------------------------------
%
%\labday{Saturday, 27 March 2010}
%
%\experiment{Bulleted list example} % You don't need to make a \newexperiment if you only plan on referencing it once
%
%This is a bulleted list:
%
%\begin{itemize}
%\item Item 1
%\item Item 2
%\item \ldots and so on
%\end{itemize}
%
%%-----------------------------------------
%
%\experiment{example}
%
%\lipsum[6]
%
%%-----------------------------------------
%
%\experiment{example2}
%
%\lipsum[7]
%
%%----------------------------------------------------------------------------------------
%%	FORMULAE AND MEDIA RECIPES
%%----------------------------------------------------------------------------------------
%
%\labday{} % We don't want a date here so we make the labday blank
%
%\begin{center}
%\HRule \\[0.4cm]
%{\huge \textbf{Formulae and Media Recipes}}\\[0.4cm] % Heading
%\HRule \\[1.5cm]
%\end{center}
%
%%----------------------------------------------------------------------------------------
%%	MEDIA RECIPES
%%----------------------------------------------------------------------------------------
%
%\newpage
%
%\huge \textbf{Media} \\ \\
%
%\normalsize \textbf{Media 1}\\
%\begin{table}[H]
%\begin{tabular}{l l l}
%\toprule
%\textbf{Compound} & \textbf{1L} & \textbf{0.5L}\\
%\toprule
%Compound 1 & 10g & 5g\\
%Compound 2 & 20g & 10g\\
%\bottomrule
%\end{tabular}
%\caption{Ingredients in Media 1.}
%\label{tab:med1}
%\end{table}
%
%%-----------------------------------------
%
%%\textbf{Media 2}\\ \\
%
%%Description
%
%%----------------------------------------------------------------------------------------
%%	FORMULAE
%%----------------------------------------------------------------------------------------
%
%\newpage
%
%\huge \textbf{Formulae} \\ \\
%
%\normalsize \textbf{Formula 1 - Pythagorean theorem}\\ \\
%$a^2 + b^2 = c^2$\\ \\
%
%%-----------------------------------------
%
%%\textbf{Formula X - Description}\\ \\
%
%%Formula

%----------------------------------------------------------------------------------------

\end{document}